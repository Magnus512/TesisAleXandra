% **************************************************
\chapter*{Introducción}
\addcontentsline{toc}{chapter}{Introducción} % --- Agregar al índice
\markboth{INTRODUCCIÓN}{INTRODUCCIÓN} % --- Encabezado 
% **************************************************
\chapterquote{Que lorem ipsum ad his scripta blandit partiendo, eum fastidii accumsan euripidis in, eum liber hendrerit an. Que lorem ipsum ad his scripta blandit partiendo, eum fastidii accumsan euripidis in, eum liber hendrerit an.}{Nombre del Autor}{Aportación, 2016}
% **************************************************
% --------------------------------------------------
\lettrine{D}{urante} el procesamiento de los cereales en grano, se generan subproductos de alto valor. Por ejemplo, el proceso de nixtamalización implica una hidrólisis alcalina que degrada y solubiliza los componentes de la pared celular del grano de maíz, que facilita la eliminación del pericarpio. Dicho subproducto contiene polisacáridos no celulósicos, principalmente arabinoxilanos ferulados. Otros subproductos del procesamiento de cereales tales como el pericarpio de trigo y el nejayote, se han estudiado como fuentes potenciales de arabinoxilanos ferulados, aunque se pueden extraer de otros materiales lignocelulósicos. Las caracteristicas moleculares y propiedades funcionales de los arabinoxilanos ferulados dependen de la fuente y del método de extracción de los mismos. La presencia del ácido ferúlico en esta molecula, le confiere propiedades antioxidantes y la capacidad de formar hidrogeles covalentes en presencia de agentes oxidantes, dichos hidrogeles poseen propiedades prebióticas, una estructura porosa, olor y sabor neutros, estabilidad al pH, a la temperatura y a cambios en la fuerza iónica. Se les atribuye gran potencial de aplicación como matrices para la liberación controlada de biomoléculas. Estas propiedades funcionales aumentan su potencial de aplicación en la industria agroalimentaria, biomédica y cosmética. 

El encapsulamiento implica la incorporación de aditivos en forma de cápsulas en el alimento o para enmascarar olores y sabores. Los materiales encapsulados pueden ser protegidos para aumentar su estabilidad y mantener su viabilidad. El uso de este proceso para adicionar edulcorantes tales como aspartamo y diferentes sabores en la goma de mascar, es bien conocido. Por otro lado, la industria de la confitería es una de las más competidas, por ello, el desarrollo y diversificación de productos es una estrategia para generar valor agregado y así mantenerse en el gusto de los consumidores; dicha estrategia también permite el ingreso de nuevos competidores al mercado. 

Dado de que en México, la industria productora de harina de maíz blanco, genera grandes cantidades de pericarpio, se requieren alternativas para su utilización. Por lo tanto, los arabinoxilanos ferulados extraídos de esta fuente residual, representan una excelente alternativa para el desarrollo de productos de confitería. Su extracción permitiría darles un uso como agente texturizante y emulsificante que permita obtener nuevas texturas; o como agente encapsulante para adicionar sabores, colores y/o ingredientes funcionales a las formulaciones.
