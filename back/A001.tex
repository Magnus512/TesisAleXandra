% **************************************************
\chapter{Ensayos químicos}\label{anexo:CurvasCalibracion}
% **************************************************
\chapterquote{Que lorem ipsum ad his scripta blandit partiendo, eum fastidii accumsan euripidis in, eum liber hendrerit an. Que lorem ipsum ad his scripta blandit partiendo, eum fastidii accumsan euripidis in, eum liber hendrerit an.}{Nombre del Autor}{Aportación, 2016}
% **************************************************
% --------------------------------------------------
\section{Determinación de proteínas}
% --------------------------------------------------
\lettrine{S}{e} realizó mediante el Método de Bradford, para ello se mezcló una alícuota de solución de MBG al 10 \% (p/v) con 50 partes del reactivo de Bradford (previamente diluido con agua en relación 1:4). El ensayo se realizó a temperatura ambiente, el desarrollo de color se inició inmediatamente y se registró su absorbancia a 595 nm. La concentración de proteína en la muestra se determinó por interpolación utilizando una curva estándar de $\beta$-Lactoglobulina.


